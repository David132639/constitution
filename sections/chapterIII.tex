\section{\underline{Chapter III - Meetings, Elections, and Constitutional Amendments}}

~

~

\large{\textbf{Article 8 - Meetings}}

~

\underline{Paragraph A} - \textbf{Executive Board Meetings} shall be
held at least bimonthly and have a minimum of five days' notice.

~

\underline{Paragraph B} - \textbf{Committee Meetings} shall be
held at as deemed necessary by the President and have a minimum of three days' notice.

~

\underline{Paragraph C} - \textbf{Annual General Meetings} (AGMs) shall:

\begin{itemize}

    \item{Take place in March every year.}

    \item{Enable members of the Executive Board to discuss the strengths, weaknesses, and future direction of The Society.}

    \item{Enable members of the Executive Board to discuss the financial, marketing, communication and project reports of The Society.}

    \item{Have fourteen days' notice given by the Secretary.}

    \item{Allow membership to make comments and suggestions. To this end, the Secretary is responsible for engaging with the membership, collecting comments and suggestions, and presenting these to the Executive Board.}

    \item{Have a quorum of one half of the Executive Board.}

\end{itemize}

~

\underline{Paragraph D} - \textbf{Extraordinary General Meetings} (EGMs)
shall:

\begin{itemize}

    \item{Be called either by the President, 50\% of the Executive Board, or by submission of a formal written request by 10\% of the membership.}

    \item{Have three days' notice given by the Secretary.}

    \item{Have a quorum of one half of the Executive Board.}

\end{itemize}

~

~

\newpage

\large{\textbf{Article 9 - Elections}}

~

\underline{Paragraph A} - Elections shall be held during the AGM, or at an EGM in
the event of a vacant position within the Executive Board.

~

\underline{Paragraph B} - Elections shall be held by a secret paper ballot or a digital equivalent. 
Proxy voting shall not be permitted.

~

\underline{Paragraph C} - Voting is open to all full members of The
Society. Associate members shall not have voting rights.

~

\underline{Paragraph D} - All candidates running for an Executive Board position
must be current Committee members. They also must fulfil any criteria set by bodies the Society is affiliated with.

~

\underline{Paragraph E} - All candidates must complete an application
form and submit it to the Executive Board at least one day before the
election.

~

\underline{Paragraph F} - In the event of a contested election, the
candidate(s) that are not elected may run for any other open position.

~

\underline{Paragraph G} - The term of office shall be two semesters for
all positions. New office bearers shall take office at the conclusion of the Annual Ball,
or on 1st June should no ball be organised or should it take place before the AGM.


~

\underline{Paragraph H} - The newly elected Executive Board members
shall be formally introduced at the annual ball.

~

\underline{Paragraph I} - Current Executive Board members may contest vacant Executive Board positions at an EGM
without forgoing their current position. Should they be successful in achieving a new position, their former position
shall become vacant, and an election for that position held immediately.

~

\newpage

\large{\textbf{Article 10 - Constitutional Amendments}}

~

\underline{Paragraph A} - Constitutional amendments can only be made in
AGMs or EGMs.

~

\underline{Paragraph B} - A constitutional amendment can only be
proposed by the submission of a formal written request by either 10\% of
the membership, 10\% of the Committee, or a majority of the Executive Board.

~

\underline{Paragraph C} - In order to pass, a constitutional amendment
must 
be approved by either: two-thirds of Executive Board members present 
and voting and majority of members present and voting, or three quarters of 
members present and voting. 

~
